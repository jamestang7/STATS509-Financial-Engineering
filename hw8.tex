% Options for packages loaded elsewhere
\PassOptionsToPackage{unicode}{hyperref}
\PassOptionsToPackage{hyphens}{url}
%
\documentclass[
]{article}
\usepackage{amsmath,amssymb}
\usepackage{lmodern}
\usepackage{iftex}
\ifPDFTeX
  \usepackage[T1]{fontenc}
  \usepackage[utf8]{inputenc}
  \usepackage{textcomp} % provide euro and other symbols
\else % if luatex or xetex
  \usepackage{unicode-math}
  \defaultfontfeatures{Scale=MatchLowercase}
  \defaultfontfeatures[\rmfamily]{Ligatures=TeX,Scale=1}
\fi
% Use upquote if available, for straight quotes in verbatim environments
\IfFileExists{upquote.sty}{\usepackage{upquote}}{}
\IfFileExists{microtype.sty}{% use microtype if available
  \usepackage[]{microtype}
  \UseMicrotypeSet[protrusion]{basicmath} % disable protrusion for tt fonts
}{}
\makeatletter
\@ifundefined{KOMAClassName}{% if non-KOMA class
  \IfFileExists{parskip.sty}{%
    \usepackage{parskip}
  }{% else
    \setlength{\parindent}{0pt}
    \setlength{\parskip}{6pt plus 2pt minus 1pt}}
}{% if KOMA class
  \KOMAoptions{parskip=half}}
\makeatother
\usepackage{xcolor}
\IfFileExists{xurl.sty}{\usepackage{xurl}}{} % add URL line breaks if available
\IfFileExists{bookmark.sty}{\usepackage{bookmark}}{\usepackage{hyperref}}
\hypersetup{
  pdftitle={hw8},
  pdfauthor={Siwei Tang},
  hidelinks,
  pdfcreator={LaTeX via pandoc}}
\urlstyle{same} % disable monospaced font for URLs
\usepackage[margin=1in]{geometry}
\usepackage{color}
\usepackage{fancyvrb}
\newcommand{\VerbBar}{|}
\newcommand{\VERB}{\Verb[commandchars=\\\{\}]}
\DefineVerbatimEnvironment{Highlighting}{Verbatim}{commandchars=\\\{\}}
% Add ',fontsize=\small' for more characters per line
\usepackage{framed}
\definecolor{shadecolor}{RGB}{248,248,248}
\newenvironment{Shaded}{\begin{snugshade}}{\end{snugshade}}
\newcommand{\AlertTok}[1]{\textcolor[rgb]{0.94,0.16,0.16}{#1}}
\newcommand{\AnnotationTok}[1]{\textcolor[rgb]{0.56,0.35,0.01}{\textbf{\textit{#1}}}}
\newcommand{\AttributeTok}[1]{\textcolor[rgb]{0.77,0.63,0.00}{#1}}
\newcommand{\BaseNTok}[1]{\textcolor[rgb]{0.00,0.00,0.81}{#1}}
\newcommand{\BuiltInTok}[1]{#1}
\newcommand{\CharTok}[1]{\textcolor[rgb]{0.31,0.60,0.02}{#1}}
\newcommand{\CommentTok}[1]{\textcolor[rgb]{0.56,0.35,0.01}{\textit{#1}}}
\newcommand{\CommentVarTok}[1]{\textcolor[rgb]{0.56,0.35,0.01}{\textbf{\textit{#1}}}}
\newcommand{\ConstantTok}[1]{\textcolor[rgb]{0.00,0.00,0.00}{#1}}
\newcommand{\ControlFlowTok}[1]{\textcolor[rgb]{0.13,0.29,0.53}{\textbf{#1}}}
\newcommand{\DataTypeTok}[1]{\textcolor[rgb]{0.13,0.29,0.53}{#1}}
\newcommand{\DecValTok}[1]{\textcolor[rgb]{0.00,0.00,0.81}{#1}}
\newcommand{\DocumentationTok}[1]{\textcolor[rgb]{0.56,0.35,0.01}{\textbf{\textit{#1}}}}
\newcommand{\ErrorTok}[1]{\textcolor[rgb]{0.64,0.00,0.00}{\textbf{#1}}}
\newcommand{\ExtensionTok}[1]{#1}
\newcommand{\FloatTok}[1]{\textcolor[rgb]{0.00,0.00,0.81}{#1}}
\newcommand{\FunctionTok}[1]{\textcolor[rgb]{0.00,0.00,0.00}{#1}}
\newcommand{\ImportTok}[1]{#1}
\newcommand{\InformationTok}[1]{\textcolor[rgb]{0.56,0.35,0.01}{\textbf{\textit{#1}}}}
\newcommand{\KeywordTok}[1]{\textcolor[rgb]{0.13,0.29,0.53}{\textbf{#1}}}
\newcommand{\NormalTok}[1]{#1}
\newcommand{\OperatorTok}[1]{\textcolor[rgb]{0.81,0.36,0.00}{\textbf{#1}}}
\newcommand{\OtherTok}[1]{\textcolor[rgb]{0.56,0.35,0.01}{#1}}
\newcommand{\PreprocessorTok}[1]{\textcolor[rgb]{0.56,0.35,0.01}{\textit{#1}}}
\newcommand{\RegionMarkerTok}[1]{#1}
\newcommand{\SpecialCharTok}[1]{\textcolor[rgb]{0.00,0.00,0.00}{#1}}
\newcommand{\SpecialStringTok}[1]{\textcolor[rgb]{0.31,0.60,0.02}{#1}}
\newcommand{\StringTok}[1]{\textcolor[rgb]{0.31,0.60,0.02}{#1}}
\newcommand{\VariableTok}[1]{\textcolor[rgb]{0.00,0.00,0.00}{#1}}
\newcommand{\VerbatimStringTok}[1]{\textcolor[rgb]{0.31,0.60,0.02}{#1}}
\newcommand{\WarningTok}[1]{\textcolor[rgb]{0.56,0.35,0.01}{\textbf{\textit{#1}}}}
\usepackage{graphicx}
\makeatletter
\def\maxwidth{\ifdim\Gin@nat@width>\linewidth\linewidth\else\Gin@nat@width\fi}
\def\maxheight{\ifdim\Gin@nat@height>\textheight\textheight\else\Gin@nat@height\fi}
\makeatother
% Scale images if necessary, so that they will not overflow the page
% margins by default, and it is still possible to overwrite the defaults
% using explicit options in \includegraphics[width, height, ...]{}
\setkeys{Gin}{width=\maxwidth,height=\maxheight,keepaspectratio}
% Set default figure placement to htbp
\makeatletter
\def\fps@figure{htbp}
\makeatother
\setlength{\emergencystretch}{3em} % prevent overfull lines
\providecommand{\tightlist}{%
  \setlength{\itemsep}{0pt}\setlength{\parskip}{0pt}}
\setcounter{secnumdepth}{-\maxdimen} % remove section numbering
\ifLuaTeX
  \usepackage{selnolig}  % disable illegal ligatures
\fi

\title{hw8}
\author{Siwei Tang}
\date{2022-03-29}

\begin{document}
\maketitle

\hypertarget{q1}{%
\section{Q1}\label{q1}}

\hypertarget{a}{%
\subsection{a}\label{a}}

\[\hat{Y}=\begin{cases}
0 & \text{ if } x= 0\\
1 & \text{ if } x= 1\\
1 & \text{ if } x= 2
\end{cases}\]

\begin{Shaded}
\begin{Highlighting}[]
\NormalTok{(}\SpecialCharTok{{-}}\DecValTok{2} \SpecialCharTok{{-}}\DecValTok{1}\NormalTok{)}\SpecialCharTok{\^{}}\DecValTok{2} \SpecialCharTok{*} \DecValTok{1}\SpecialCharTok{/}\DecValTok{6} \SpecialCharTok{+}\NormalTok{ (}\DecValTok{2{-}1}\NormalTok{)}\SpecialCharTok{\^{}}\DecValTok{2}\SpecialCharTok{*}\DecValTok{1}\SpecialCharTok{/}\DecValTok{2}
\end{Highlighting}
\end{Shaded}

\begin{verbatim}
## [1] 2
\end{verbatim}

MSPE \(=\mathbb{E}((Y-\hat{Y})^2) = 2\)

\hypertarget{b}{%
\subsection{b}\label{b}}

\begin{Shaded}
\begin{Highlighting}[]
\NormalTok{muy }\OtherTok{=} \FunctionTok{c}\NormalTok{(}\SpecialCharTok{{-}}\DecValTok{2}\NormalTok{,}\DecValTok{0}\NormalTok{,}\DecValTok{1}\NormalTok{,}\DecValTok{2}\NormalTok{) }\SpecialCharTok{\%*\%} \FunctionTok{c}\NormalTok{(}\DecValTok{1}\SpecialCharTok{/}\DecValTok{6}\NormalTok{,}\DecValTok{1}\SpecialCharTok{/}\DecValTok{6}\NormalTok{,}\DecValTok{1}\SpecialCharTok{/}\DecValTok{6}\NormalTok{,}\DecValTok{1}\SpecialCharTok{/}\DecValTok{2}\NormalTok{)}
\NormalTok{mux }\OtherTok{=} \FunctionTok{c}\NormalTok{(}\DecValTok{2}\NormalTok{,}\DecValTok{0}\NormalTok{,}\DecValTok{1}\NormalTok{,}\DecValTok{2}\NormalTok{) }\SpecialCharTok{\%*\%} \FunctionTok{c}\NormalTok{(}\DecValTok{1}\SpecialCharTok{/}\DecValTok{6}\NormalTok{,}\DecValTok{1}\SpecialCharTok{/}\DecValTok{6}\NormalTok{,}\DecValTok{1}\SpecialCharTok{/}\DecValTok{6}\NormalTok{,}\DecValTok{1}\SpecialCharTok{/}\DecValTok{2}\NormalTok{)}
\NormalTok{varx }\OtherTok{=} \FunctionTok{c}\NormalTok{(}\DecValTok{4}\NormalTok{,}\DecValTok{0}\NormalTok{,}\DecValTok{1}\NormalTok{,}\DecValTok{4}\NormalTok{) }\SpecialCharTok{\%*\%} \FunctionTok{c}\NormalTok{(}\DecValTok{1}\SpecialCharTok{/}\DecValTok{6}\NormalTok{,}\DecValTok{1}\SpecialCharTok{/}\DecValTok{6}\NormalTok{,}\DecValTok{1}\SpecialCharTok{/}\DecValTok{6}\NormalTok{,}\DecValTok{1}\SpecialCharTok{/}\DecValTok{2}\NormalTok{) }\SpecialCharTok{{-}}\NormalTok{ mux}\SpecialCharTok{\^{}}\DecValTok{2}
\CommentTok{\# when y \textgreater{}= 0}
\NormalTok{covxy }\OtherTok{=} \FunctionTok{c}\NormalTok{(}\DecValTok{2}\SpecialCharTok{*{-}}\DecValTok{2}\NormalTok{,}\DecValTok{0}\NormalTok{,}\DecValTok{1}\NormalTok{,}\DecValTok{4}\NormalTok{) }\SpecialCharTok{\%*\%} \FunctionTok{c}\NormalTok{(}\DecValTok{1}\SpecialCharTok{/}\DecValTok{6}\NormalTok{,}\DecValTok{1}\SpecialCharTok{/}\DecValTok{6}\NormalTok{,}\DecValTok{1}\SpecialCharTok{/}\DecValTok{6}\NormalTok{,}\DecValTok{1}\SpecialCharTok{/}\DecValTok{2}\NormalTok{) }\SpecialCharTok{{-}}\NormalTok{ muy }\SpecialCharTok{*}\NormalTok{ mux}
\FunctionTok{cat}\NormalTok{(}\StringTok{"mux is:"}\NormalTok{,mux,}\StringTok{";muy:"}\NormalTok{,muy,}\StringTok{"cov(x,y):"}\NormalTok{,covxy,}\StringTok{\textquotesingle{}}\SpecialCharTok{\textbackslash{}n}\StringTok{\textquotesingle{}}\NormalTok{)}
\end{Highlighting}
\end{Shaded}

\begin{verbatim}
## mux is: 1.5 ;muy: 0.8333333 cov(x,y): 0.25
\end{verbatim}

\begin{Shaded}
\begin{Highlighting}[]
\FunctionTok{cat}\NormalTok{(}\StringTok{"Y hat is "}\NormalTok{ ,muy }\SpecialCharTok{+}\NormalTok{ (covxy }\SpecialCharTok{*}\NormalTok{ (}\SpecialCharTok{{-}}\NormalTok{mux)}\SpecialCharTok{/}\NormalTok{varx), }\StringTok{"+"}\NormalTok{,covxy}\SpecialCharTok{/}\NormalTok{varx,}\StringTok{"X"}\NormalTok{)}
\end{Highlighting}
\end{Shaded}

\begin{verbatim}
## Y hat is  0.1904762 + 0.4285714 X
\end{verbatim}

\begin{Shaded}
\begin{Highlighting}[]
\NormalTok{yhat2 }\OtherTok{=} \FunctionTok{as.vector}\NormalTok{(muy }\SpecialCharTok{+}\NormalTok{ (covxy }\SpecialCharTok{*}\NormalTok{ (}\SpecialCharTok{{-}}\NormalTok{mux)}\SpecialCharTok{/}\NormalTok{varx) }\SpecialCharTok{+}\NormalTok{ covxy}\SpecialCharTok{/}\NormalTok{varx}\SpecialCharTok{*}\FunctionTok{c}\NormalTok{(}\DecValTok{0}\NormalTok{,}\DecValTok{1}\NormalTok{,}\DecValTok{2}\NormalTok{))}
\end{Highlighting}
\end{Shaded}

\begin{verbatim}
## Warning in covxy/varx * c(0, 1, 2): Recycling array of length 1 in array-vector arithmetic is deprecated.
##   Use c() or as.vector() instead.
\end{verbatim}

\begin{verbatim}
## Warning in muy + (covxy * (-mux)/varx) + covxy/varx * c(0, 1, 2): Recycling array of length 1 in array-vector arithmetic is deprecated.
##   Use c() or as.vector() instead.
\end{verbatim}

\begin{Shaded}
\begin{Highlighting}[]
\NormalTok{yhat2 }
\end{Highlighting}
\end{Shaded}

\begin{verbatim}
## [1] 0.1904762 0.6190476 1.0476190
\end{verbatim}

\(\hat{Y} = \mu_Y Y+\Sigma(Y,X)\Sigma(X)^{-1}(X-\mu_X)\)\textbackslash{}
\(\hat{Y} = 0.1904762 + 0.4285714 X\) \[\hat{Y}=\begin{cases}
0.1904762  & \text{ if } x= 0\\
0.6190476 & \text{ if } x= 1\\
1.0476190 & \text{ if } x= 2
\end{cases}\]

\begin{Shaded}
\begin{Highlighting}[]
\CommentTok{\# calculating MSPE}
\NormalTok{mspe }\OtherTok{=} \FunctionTok{c}\NormalTok{(}\FloatTok{0.1904762}\SpecialCharTok{\^{}}\DecValTok{2}\NormalTok{, (}\DecValTok{1}\SpecialCharTok{{-}} \FloatTok{0.6190476}\NormalTok{)}\SpecialCharTok{\^{}}\DecValTok{2}\NormalTok{, (}\DecValTok{2} \SpecialCharTok{{-}} \FloatTok{1.0476190}\NormalTok{)}\SpecialCharTok{\^{}}\DecValTok{2}\NormalTok{,}
\NormalTok{         (}\SpecialCharTok{{-}}\DecValTok{2} \SpecialCharTok{{-}} \FloatTok{1.0476190}\NormalTok{)}\SpecialCharTok{\^{}}\DecValTok{2}\NormalTok{) }\SpecialCharTok{\%*\%} \FunctionTok{c}\NormalTok{(}\DecValTok{1}\SpecialCharTok{/}\DecValTok{6}\NormalTok{,}\DecValTok{1}\SpecialCharTok{/}\DecValTok{6}\NormalTok{,}\DecValTok{1}\SpecialCharTok{/}\DecValTok{2}\NormalTok{,}\DecValTok{1}\SpecialCharTok{/}\DecValTok{6}\NormalTok{)}
\FunctionTok{cat}\NormalTok{(}\StringTok{"MSPE:"}\NormalTok{,mspe)}
\end{Highlighting}
\end{Shaded}

\begin{verbatim}
## MSPE: 2.031746
\end{verbatim}

\hypertarget{c}{%
\subsection{c}\label{c}}

Because the MSPE is optimal when \(\mu_x = \mu_y = 0\). We didn't center
R.V. X and Y.

\hypertarget{q2}{%
\section{Q2}\label{q2}}

\hypertarget{a-1}{%
\subsection{a}\label{a-1}}

\(\gamma(k) = cov(Y_t,Y_{t-k}) = cov(\phi_1(Y_{t-1}-\mu)+\phi_2(Y_{t-2}-\mu)+\epsilon_t,Y_{t-k})\\\)
\(= cov(\phi_1Y_{t-1},Y_{t-k})+cov(\phi_2Y_{t-2},Y_{t-k}) + 0\\\)
\(= \phi_1\gamma(k-1) + \phi_2\gamma(k-2)\\\)

Divide \(\gamma(0)\) on both left side and right side, gives us

\(\rho(k) = \phi_1\rho(k-1) + \phi_2\rho(k-2)\\ \square\)

\hypertarget{b-1}{%
\subsection{b}\label{b-1}}

Because \(\rho(0) = 1\) for time series sequence \(Y_t, \forall t\)
Applying the result when k =1 and 2 gives us
\(\rho(1) = \rho(0)\phi_1 +\rho(1)\phi_2\)
\(\rho(2) = \rho(1)\phi_1 +\rho(0)\phi_2\) Therefore (\(\phi_1,\phi_2\))
solves this linear system. \(\square\) \#\# c

\begin{Shaded}
\begin{Highlighting}[]
\NormalTok{rho }\OtherTok{=} \FunctionTok{matrix}\NormalTok{(}\FunctionTok{c}\NormalTok{(.}\DecValTok{3}\NormalTok{,.}\DecValTok{2}\NormalTok{),}\AttributeTok{nrow =} \DecValTok{2}\NormalTok{, }\AttributeTok{ncol =}\DecValTok{1}\NormalTok{)}
\NormalTok{sqr\_mtx }\OtherTok{=} \FunctionTok{matrix}\NormalTok{(}\FunctionTok{c}\NormalTok{(}\DecValTok{1}\NormalTok{,.}\DecValTok{3}\NormalTok{,.}\DecValTok{3}\NormalTok{,}\DecValTok{1}\NormalTok{), }\DecValTok{2}\NormalTok{,}\DecValTok{2}\NormalTok{)}
\NormalTok{res }\OtherTok{=} \FunctionTok{solve}\NormalTok{(sqr\_mtx) }\SpecialCharTok{\%*\%}\NormalTok{ rho}
\FunctionTok{print}\NormalTok{(res)}
\end{Highlighting}
\end{Shaded}

\begin{verbatim}
##           [,1]
## [1,] 0.2637363
## [2,] 0.1208791
\end{verbatim}

\begin{Shaded}
\begin{Highlighting}[]
\NormalTok{phi1 }\OtherTok{=}\NormalTok{ res[}\DecValTok{1}\NormalTok{]}
\NormalTok{phi2 }\OtherTok{=}\NormalTok{ res[}\DecValTok{2}\NormalTok{]}
\NormalTok{phi3 }\OtherTok{=} \FunctionTok{c}\NormalTok{(.}\DecValTok{2}\NormalTok{,.}\DecValTok{3}\NormalTok{) }\SpecialCharTok{\%*\%}\NormalTok{ res}
\FunctionTok{cat}\NormalTok{(}\StringTok{"Phi1 is"}\NormalTok{,phi1,}\StringTok{"}\SpecialCharTok{\textbackslash{}n}\StringTok{Phi2 is"}\NormalTok{,phi2,}\StringTok{"}\SpecialCharTok{\textbackslash{}n}\StringTok{"}\NormalTok{)}
\end{Highlighting}
\end{Shaded}

\begin{verbatim}
## Phi1 is 0.2637363 
## Phi2 is 0.1208791
\end{verbatim}

\begin{Shaded}
\begin{Highlighting}[]
\FunctionTok{cat}\NormalTok{(}\StringTok{"Phi3 is"}\NormalTok{,phi3)}
\end{Highlighting}
\end{Shaded}

\begin{verbatim}
## Phi3 is 0.08901099
\end{verbatim}

\hypertarget{q3}{%
\section{Q3}\label{q3}}

\hypertarget{section}{%
\subsection{1}\label{section}}

\begin{Shaded}
\begin{Highlighting}[]
\FunctionTok{library}\NormalTok{(stats)}
\FunctionTok{library}\NormalTok{(ggplot2)}
\FunctionTok{library}\NormalTok{(forecast)}
\end{Highlighting}
\end{Shaded}

\begin{verbatim}
## Registered S3 method overwritten by 'quantmod':
##   method            from
##   as.zoo.data.frame zoo
\end{verbatim}

\begin{Shaded}
\begin{Highlighting}[]
\FunctionTok{library}\NormalTok{(FitAR)}
\end{Highlighting}
\end{Shaded}

\begin{verbatim}
## Loading required package: lattice
\end{verbatim}

\begin{verbatim}
## Loading required package: leaps
\end{verbatim}

\begin{verbatim}
## Loading required package: ltsa
\end{verbatim}

\begin{verbatim}
## Loading required package: bestglm
\end{verbatim}

\begin{verbatim}
## 
## Attaching package: 'FitAR'
\end{verbatim}

\begin{verbatim}
## The following object is masked from 'package:forecast':
## 
##     BoxCox
\end{verbatim}

\begin{Shaded}
\begin{Highlighting}[]
\NormalTok{root }\OtherTok{=} \StringTok{\textquotesingle{}./datasets/\textquotesingle{}}
\NormalTok{file }\OtherTok{=} \FunctionTok{paste}\NormalTok{(root,}\StringTok{\textquotesingle{}RUT\_03\_2015{-}03\_2019.csv\textquotesingle{}}\NormalTok{, }\AttributeTok{sep =}\StringTok{""}\NormalTok{)}
\NormalTok{df }\OtherTok{=} \FunctionTok{read.csv}\NormalTok{(file)}
\NormalTok{n }\OtherTok{=} \FunctionTok{dim}\NormalTok{(df)[}\DecValTok{1}\NormalTok{]}
\NormalTok{r }\OtherTok{=}\NormalTok{ df}\SpecialCharTok{$}\NormalTok{Adj.Close[}\SpecialCharTok{{-}}\DecValTok{1}\NormalTok{]}\SpecialCharTok{/}\NormalTok{df}\SpecialCharTok{$}\NormalTok{Adj.Close[}\SpecialCharTok{{-}}\NormalTok{n] }\SpecialCharTok{{-}} \DecValTok{1}
\FunctionTok{acf}\NormalTok{(}\FunctionTok{as.vector}\NormalTok{(r), }\AttributeTok{lag =}\DecValTok{20}\NormalTok{)}
\end{Highlighting}
\end{Shaded}

\includegraphics{hw8_files/figure-latex/unnamed-chunk-5-1.pdf}

\begin{Shaded}
\begin{Highlighting}[]
\CommentTok{\# box{-}Ljng test}
\FunctionTok{Box.test}\NormalTok{(r, }\AttributeTok{lag =} \DecValTok{10}\NormalTok{, }\AttributeTok{type =} \StringTok{"Ljung{-}Box"}\NormalTok{)}
\end{Highlighting}
\end{Shaded}

\begin{verbatim}
## 
##  Box-Ljung test
## 
## data:  r
## X-squared = 11.605, df = 10, p-value = 0.3124
\end{verbatim}

The p-value of Box-Ljung test is 0.3124, which indicates that we cannot
reject the null hypothesis. So we have accept the null that the auto
correlation are all zeros out to lag 10. From the ACF plot, we dont see
any big violations of autocorrelation within the 95\% confidence
interval till lag = 20. This shows that there is no strong
auto-correlation on the sequence. \#\# b

\begin{Shaded}
\begin{Highlighting}[]
\NormalTok{RUT\_ar }\OtherTok{=} \FunctionTok{ar}\NormalTok{(r, }\AttributeTok{aic=} \ConstantTok{TRUE}\NormalTok{, }\AttributeTok{order.max =} \DecValTok{6}\NormalTok{, }\AttributeTok{method =} \StringTok{"mle"}\NormalTok{)}
\NormalTok{RUT\_ar}
\end{Highlighting}
\end{Shaded}

\begin{verbatim}
## 
## Call:
## ar(x = r, aic = TRUE, order.max = 6, method = "mle")
## 
## Coefficients:
##       1        2        3        4  
##  0.0144  -0.0050  -0.1422   0.1489  
## 
## Order selected 4  sigma^2 estimated as  0.0005315
\end{verbatim}

\begin{Shaded}
\begin{Highlighting}[]
\CommentTok{\# fit the model with ARIMA}
\NormalTok{RUT\_arima }\OtherTok{=} \FunctionTok{arima}\NormalTok{(r, }\AttributeTok{order =} \FunctionTok{c}\NormalTok{(}\DecValTok{4}\NormalTok{,}\DecValTok{0}\NormalTok{,}\DecValTok{0}\NormalTok{),}
                  \AttributeTok{method =} \StringTok{"ML"}\NormalTok{)}
\NormalTok{temp\_df }\OtherTok{=} \FunctionTok{data.frame}\NormalTok{(}\AttributeTok{X =}\NormalTok{ RUT\_arima}\SpecialCharTok{$}\NormalTok{residuals)}
\NormalTok{p }\OtherTok{\textless{}{-}} \FunctionTok{ggplot}\NormalTok{(temp\_df, }\FunctionTok{aes}\NormalTok{(}\AttributeTok{sample =}\NormalTok{ X))}
\NormalTok{p }\SpecialCharTok{+} \FunctionTok{stat\_qq}\NormalTok{() }\SpecialCharTok{+} 
  \FunctionTok{stat\_qq\_line}\NormalTok{()}\SpecialCharTok{+}
  \FunctionTok{labs}\NormalTok{(}\AttributeTok{title =} \StringTok{"QQ plot of reisiduals"}\NormalTok{,}
       \AttributeTok{x =} \StringTok{"Theoretical quantile"}\NormalTok{,}
       \AttributeTok{y =} \StringTok{"Sample quantile"}\NormalTok{)}
\end{Highlighting}
\end{Shaded}

\begin{verbatim}
## Don't know how to automatically pick scale for object of type ts. Defaulting to continuous.
## Don't know how to automatically pick scale for object of type ts. Defaulting to continuous.
\end{verbatim}

\includegraphics{hw8_files/figure-latex/unnamed-chunk-6-1.pdf}

\begin{Shaded}
\begin{Highlighting}[]
\CommentTok{\# Box\_Ljung test}
\FunctionTok{Box.test}\NormalTok{(}\FunctionTok{residuals}\NormalTok{(RUT\_arima), }\AttributeTok{lag =} \DecValTok{10}\NormalTok{, }\AttributeTok{type =} \StringTok{"Ljung{-}Box"}\NormalTok{, }\AttributeTok{fitdf =} \DecValTok{4}\NormalTok{)}
\end{Highlighting}
\end{Shaded}

\begin{verbatim}
## 
##  Box-Ljung test
## 
## data:  residuals(RUT_arima)
## X-squared = 2.8779, df = 6, p-value = 0.824
\end{verbatim}

\begin{Shaded}
\begin{Highlighting}[]
\FunctionTok{acf}\NormalTok{(}\FunctionTok{as.vector}\NormalTok{(}\FunctionTok{residuals}\NormalTok{(RUT\_arima)), }\AttributeTok{lag =}\DecValTok{10}\NormalTok{, }\AttributeTok{main =} \StringTok{"ACF of Residuals"}\NormalTok{)}
\end{Highlighting}
\end{Shaded}

\includegraphics{hw8_files/figure-latex/unnamed-chunk-6-2.pdf}

\begin{Shaded}
\begin{Highlighting}[]
\FunctionTok{LBQPlot}\NormalTok{(}\FunctionTok{residuals}\NormalTok{(RUT\_arima))}
\end{Highlighting}
\end{Shaded}

\includegraphics{hw8_files/figure-latex/unnamed-chunk-6-3.pdf}

\begin{Shaded}
\begin{Highlighting}[]
\FunctionTok{checkresiduals}\NormalTok{(RUT\_arima)}
\end{Highlighting}
\end{Shaded}

\includegraphics{hw8_files/figure-latex/unnamed-chunk-6-4.pdf}

\begin{verbatim}
## 
##  Ljung-Box test
## 
## data:  Residuals from ARIMA(4,0,0) with non-zero mean
## Q* = 2.8779, df = 5, p-value = 0.7188
## 
## Model df: 5.   Total lags used: 10
\end{verbatim}

\begin{Shaded}
\begin{Highlighting}[]
\CommentTok{\# auto.arima(r, max.p = 20,max.q = 0,d=0 ,ic = "aic")}
\end{Highlighting}
\end{Shaded}

The AR model picks the order \texttt{p\ =\ 4}. Numerical test of
Box-Ljung indicates that with order p = 4, lag = 10 , there is no
auto-correlation in the residuals. Graphic plot on Box-Ljung test shows
that all p-values are above the 95\% interval, which means there is no
correlation left in the model.The QQ plot on the residuals shows a bad
approximation on its normality, as most of the points locate along the
line. But its worth noticing that the residuals are heavy tailed, which
means it has more density on its tails.

\hypertarget{c-1}{%
\subsection{c}\label{c-1}}

\begin{Shaded}
\begin{Highlighting}[]
\FunctionTok{library}\NormalTok{(tseries)}
\NormalTok{f }\OtherTok{=} \FunctionTok{forecast}\NormalTok{(RUT\_arima, }\AttributeTok{h =}\DecValTok{12}\NormalTok{)}
\NormalTok{n }\OtherTok{=} \FunctionTok{dim}\NormalTok{(df)[}\DecValTok{1}\NormalTok{]}
\NormalTok{new\_date }\OtherTok{=} \FunctionTok{seq.Date}\NormalTok{(}\FunctionTok{as.Date}\NormalTok{(df}\SpecialCharTok{$}\NormalTok{Date)[n], }\AttributeTok{length =} \DecValTok{12}\NormalTok{, }\AttributeTok{by =} \StringTok{"week"}\NormalTok{)}
\NormalTok{upper }\OtherTok{=}\NormalTok{ f}\SpecialCharTok{$}\NormalTok{upper[}\DecValTok{13}\SpecialCharTok{:}\DecValTok{24}\NormalTok{]}
\NormalTok{lower }\OtherTok{=}\NormalTok{ f}\SpecialCharTok{$}\NormalTok{lower[}\DecValTok{13}\SpecialCharTok{:}\DecValTok{24}\NormalTok{]}
\NormalTok{temp }\OtherTok{=} \FunctionTok{data.frame}\NormalTok{(}\AttributeTok{Date =}\NormalTok{ new\_date, }\AttributeTok{lower =}\NormalTok{ lower, }\AttributeTok{upper =}\NormalTok{ upper, }\AttributeTok{y =}\NormalTok{ f}\SpecialCharTok{$}\NormalTok{mean)}
\NormalTok{new\_date }\OtherTok{=} \FunctionTok{append}\NormalTok{(df}\SpecialCharTok{$}\NormalTok{Date,}\FunctionTok{as.character}\NormalTok{(new\_date))[}\SpecialCharTok{{-}}\DecValTok{1}\NormalTok{]}
\NormalTok{new\_return }\OtherTok{=} \FunctionTok{c}\NormalTok{(r, f}\SpecialCharTok{$}\NormalTok{mean)}
\NormalTok{new\_df }\OtherTok{=} \FunctionTok{data.frame}\NormalTok{(}\AttributeTok{Date =}\NormalTok{ new\_date, }\AttributeTok{y =}\NormalTok{ new\_return)}
\NormalTok{p }\OtherTok{=} \FunctionTok{ggplot}\NormalTok{(new\_df) }\SpecialCharTok{+}
  \FunctionTok{geom\_line}\NormalTok{(}\FunctionTok{aes}\NormalTok{(}\FunctionTok{as.Date}\NormalTok{(Date), y),}
            \AttributeTok{color =} \StringTok{"black"}\NormalTok{,}
            \AttributeTok{size  =} \FloatTok{0.5}\NormalTok{)}\SpecialCharTok{+}
  \FunctionTok{geom\_line}\NormalTok{(}\AttributeTok{data =}\NormalTok{ temp,}
                          \FunctionTok{aes}\NormalTok{(Date, y),}
                          \AttributeTok{color =} \StringTok{"red"}\NormalTok{,}
                          \AttributeTok{size  =} \DecValTok{1}\NormalTok{) }\SpecialCharTok{+} 
  \FunctionTok{geom\_ribbon}\NormalTok{(}\FunctionTok{aes}\NormalTok{(}\AttributeTok{x =} \FunctionTok{as.Date}\NormalTok{(Date), }\AttributeTok{ymin =} \FunctionTok{c}\NormalTok{(r,lower), }
                  \AttributeTok{ymax =} \FunctionTok{c}\NormalTok{(r, upper)),}
              \AttributeTok{fill =} \StringTok{"grey70"}\NormalTok{, }\AttributeTok{alpha =} \FloatTok{0.5}\NormalTok{)}\SpecialCharTok{+}
  \FunctionTok{labs}\NormalTok{(}\AttributeTok{title =} \StringTok{"AR p = 4 Prediction on Return"}\NormalTok{,}
       \AttributeTok{x =} \StringTok{"Date"}\NormalTok{,}
       \AttributeTok{y =} \StringTok{"Return"}\NormalTok{)}
\NormalTok{p}
\end{Highlighting}
\end{Shaded}

\includegraphics{hw8_files/figure-latex/unnamed-chunk-7-1.pdf}

\begin{Shaded}
\begin{Highlighting}[]
\FunctionTok{print}\NormalTok{(temp)}
\end{Highlighting}
\end{Shaded}

\begin{verbatim}
##          Date       lower      upper             y
## 1  2019-03-25 -0.05296470 0.03740474 -7.779978e-03
## 2  2019-04-01 -0.03668235 0.05369651  8.507080e-03
## 3  2019-04-08 -0.04991637 0.04046352 -4.726424e-03
## 4  2019-04-15 -0.04181967 0.04947072  3.825525e-03
## 5  2019-04-22 -0.04718444 0.04503965 -1.072395e-03
## 6  2019-04-29 -0.04299131 0.04923419  3.121442e-03
## 7  2019-05-06 -0.04610103 0.04614025  1.960837e-05
## 8  2019-05-13 -0.04423487 0.04808279  1.923958e-03
## 9  2019-05-20 -0.04552656 0.04680893  6.411867e-04
## 10 2019-05-27 -0.04448900 0.04784657  1.678784e-03
## 11 2019-06-03 -0.04520198 0.04713686  9.674387e-04
## 12 2019-06-10 -0.04475318 0.04758917  1.417997e-03
\end{verbatim}

\hypertarget{c-2}{%
\subsection{c}\label{c-2}}

\begin{Shaded}
\begin{Highlighting}[]
\FunctionTok{arma}\NormalTok{(r)}
\end{Highlighting}
\end{Shaded}

\begin{verbatim}
## 
## Call:
## arma(x = r)
## 
## Coefficient(s):
##       ar1        ma1  intercept  
## -0.821579   0.760840   0.002586
\end{verbatim}

\begin{Shaded}
\begin{Highlighting}[]
\NormalTok{arma\_fit }\OtherTok{=} \FunctionTok{arima}\NormalTok{(r, }\AttributeTok{order =} \FunctionTok{c}\NormalTok{(}\DecValTok{1}\NormalTok{,}\DecValTok{1}\NormalTok{,}\DecValTok{0}\NormalTok{),}\AttributeTok{method =} \StringTok{"ML"}\NormalTok{)}
\FunctionTok{checkresiduals}\NormalTok{(arma\_fit)}
\end{Highlighting}
\end{Shaded}

\includegraphics{hw8_files/figure-latex/unnamed-chunk-8-1.pdf}

\begin{verbatim}
## 
##  Ljung-Box test
## 
## data:  Residuals from ARIMA(1,1,0)
## Q* = 46.243, df = 9, p-value = 5.427e-07
## 
## Model df: 1.   Total lags used: 10
\end{verbatim}

\begin{Shaded}
\begin{Highlighting}[]
\FunctionTok{Box.test}\NormalTok{(}\FunctionTok{residuals}\NormalTok{(arma\_fit), }\AttributeTok{lag =} \DecValTok{10}\NormalTok{, }\AttributeTok{type =} \StringTok{"Ljung{-}Box"}\NormalTok{, }\AttributeTok{fitdf =} \DecValTok{2}\NormalTok{)}
\end{Highlighting}
\end{Shaded}

\begin{verbatim}
## 
##  Box-Ljung test
## 
## data:  residuals(arma_fit)
## X-squared = 46.243, df = 8, p-value = 2.137e-07
\end{verbatim}

\begin{Shaded}
\begin{Highlighting}[]
\FunctionTok{LBQPlot}\NormalTok{(}\FunctionTok{residuals}\NormalTok{(arma\_fit))}
\end{Highlighting}
\end{Shaded}

\includegraphics{hw8_files/figure-latex/unnamed-chunk-8-2.pdf}
\texttt{Arma} reuslt shows that the best pick for order is
\texttt{p\ =\ 1,\ q\ =\ 1}, and the residuals plot implies a bad fit
because most of the points are below the 95\% interval. In addition, the
ACF plot shows strong auto-correlation when lag is small. Residuals
doesn't have normalty as well, it's quite right skewed. In summary, ARMA
model is not a good fit. \#\# e

\begin{Shaded}
\begin{Highlighting}[]
\FunctionTok{library}\NormalTok{(tseries)}
\NormalTok{f }\OtherTok{=} \FunctionTok{forecast}\NormalTok{(arma\_fit, }\AttributeTok{h =}\DecValTok{12}\NormalTok{)}
\NormalTok{n }\OtherTok{=} \FunctionTok{dim}\NormalTok{(df)[}\DecValTok{1}\NormalTok{]}
\NormalTok{new\_date }\OtherTok{=} \FunctionTok{seq.Date}\NormalTok{(}\FunctionTok{as.Date}\NormalTok{(df}\SpecialCharTok{$}\NormalTok{Date)[n], }\AttributeTok{length =} \DecValTok{12}\NormalTok{, }\AttributeTok{by =} \StringTok{"week"}\NormalTok{)}
\NormalTok{upper }\OtherTok{=}\NormalTok{ f}\SpecialCharTok{$}\NormalTok{upper[}\DecValTok{13}\SpecialCharTok{:}\DecValTok{24}\NormalTok{]}
\NormalTok{lower }\OtherTok{=}\NormalTok{ f}\SpecialCharTok{$}\NormalTok{lower[}\DecValTok{13}\SpecialCharTok{:}\DecValTok{24}\NormalTok{]}
\NormalTok{temp }\OtherTok{=} \FunctionTok{data.frame}\NormalTok{(}\AttributeTok{Date =}\NormalTok{ new\_date, }\AttributeTok{lower =}\NormalTok{ lower, }\AttributeTok{upper =}\NormalTok{ upper, }\AttributeTok{y =}\NormalTok{ f}\SpecialCharTok{$}\NormalTok{mean)}
\NormalTok{new\_date }\OtherTok{=} \FunctionTok{append}\NormalTok{(df}\SpecialCharTok{$}\NormalTok{Date,}\FunctionTok{as.character}\NormalTok{(new\_date))[}\SpecialCharTok{{-}}\DecValTok{1}\NormalTok{]}
\NormalTok{new\_return }\OtherTok{=} \FunctionTok{c}\NormalTok{(r, f}\SpecialCharTok{$}\NormalTok{mean)}
\NormalTok{new\_df }\OtherTok{=} \FunctionTok{data.frame}\NormalTok{(}\AttributeTok{Date =}\NormalTok{ new\_date, }\AttributeTok{y =}\NormalTok{ new\_return)}
\NormalTok{p }\OtherTok{=} \FunctionTok{ggplot}\NormalTok{(new\_df) }\SpecialCharTok{+}
  \FunctionTok{geom\_line}\NormalTok{(}\FunctionTok{aes}\NormalTok{(}\FunctionTok{as.Date}\NormalTok{(Date), y),}
            \AttributeTok{color =} \StringTok{"black"}\NormalTok{,}
            \AttributeTok{size  =} \FloatTok{0.5}\NormalTok{)}\SpecialCharTok{+}
  \FunctionTok{geom\_line}\NormalTok{(}\AttributeTok{data =}\NormalTok{ temp,}
                          \FunctionTok{aes}\NormalTok{(Date, y),}
                          \AttributeTok{color =} \StringTok{"red"}\NormalTok{,}
                          \AttributeTok{size  =} \DecValTok{1}\NormalTok{) }\SpecialCharTok{+} 
  \FunctionTok{geom\_ribbon}\NormalTok{(}\FunctionTok{aes}\NormalTok{(}\AttributeTok{x =} \FunctionTok{as.Date}\NormalTok{(Date), }\AttributeTok{ymin =} \FunctionTok{c}\NormalTok{(r,lower), }
                  \AttributeTok{ymax =} \FunctionTok{c}\NormalTok{(r, upper)),}
              \AttributeTok{fill =} \StringTok{"grey70"}\NormalTok{, }\AttributeTok{alpha =} \FloatTok{0.5}\NormalTok{)}\SpecialCharTok{+}
  \FunctionTok{labs}\NormalTok{(}\AttributeTok{title =} \StringTok{"AR p = 4 Prediction on Return"}\NormalTok{,}
       \AttributeTok{x =} \StringTok{"Date"}\NormalTok{,}
       \AttributeTok{y =} \StringTok{"Return"}\NormalTok{)}
\NormalTok{p}
\end{Highlighting}
\end{Shaded}

\includegraphics{hw8_files/figure-latex/unnamed-chunk-9-1.pdf}

\begin{Shaded}
\begin{Highlighting}[]
\FunctionTok{print}\NormalTok{(temp)}
\end{Highlighting}
\end{Shaded}

\begin{verbatim}
##          Date       lower      upper            y
## 1  2019-03-25 -0.06667887 0.04701678 -0.009831045
## 2  2019-04-01 -0.06314697 0.06406681  0.000459921
## 3  2019-04-08 -0.08124034 0.07190889 -0.004665724
## 4  2019-04-15 -0.08656327 0.08233770 -0.002112782
## 5  2019-04-22 -0.09645159 0.08968293 -0.003384332
## 6  2019-04-29 -0.10303999 0.09753798 -0.002751008
## 7  2019-05-06 -0.11039707 0.10426417 -0.003066449
## 8  2019-05-13 -0.11670290 0.11088422 -0.002909337
## 9  2019-05-20 -0.12296430 0.11698912 -0.002987590
## 10 2019-05-27 -0.12877272 0.12287549 -0.002948614
## 11 2019-06-03 -0.13439501 0.12845896 -0.002968027
## 12 2019-06-10 -0.13975156 0.13383485 -0.002958358
\end{verbatim}

\end{document}
